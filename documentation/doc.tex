\documentclass[11pt]{article}

% Paquetes
\usepackage[utf8]{inputenc}
\usepackage[T1]{fontenc}
\usepackage[spanish]{babel}
\usepackage{hyperref}
\usepackage{graphicx}
\usepackage{booktabs}
\usepackage{longtable}
\usepackage{enumitem}
\usepackage{geometry}
\geometry{margin=1in}
\usepackage{tabularx}
\usepackage{array}

% Información del documento
\title{Documentación del Proyecto CalificaUPV}
\author{Ricardo E. Uriegas Ibarra - 2230122}
\date{\today}

\begin{document}

\maketitle
\tableofcontents
\newpage

\section{Descripción General}

\textbf{Objetivo:} \\
Desarrollar una aplicación web en la que los alumnos de la UPV puedan calificar de forma anónima a sus profesores, utilizando una escala de 1 a 5 estrellas. La aplicación permitirá recopilar las calificaciones para generar estadísticas y ayudar a mejorar la calidad de la enseñanza.

\bigskip
\textbf{Público objetivo:} \\
Estudiantes de la UPV que deseen expresar su opinión sobre la labor de sus profesores de manera anónima.

\section{Pantallas (Wireframes/Mockups)}

Debido a la naturaleza del proyecto, se han definido las siguientes pantallas:

\subsection{Pantalla de Inicio}
\begin{itemize}
    \item Breve introducción al funcionamiento de la aplicación.
    \item Acceso a la sección de calificaciones sin necesidad de registro (la identidad permanece anónima).
\end{itemize}

\subsection{Pantalla de Listado de Profesores}
\begin{itemize}
    \item Visualización de una lista de profesores disponibles para calificar.
    \item Filtros de búsqueda (por departamento, nombre, etc.).
\end{itemize}

\subsection{Pantalla de Detalle y Calificación del Profesor}
\begin{itemize}
    \item Información básica del profesor (nombre, departamento, foto, etc.).
    \item Sistema de calificación: opción para seleccionar entre 1 y 5 estrellas.
    \item Opción para dejar un comentario opcional.
    \item Botón para enviar la calificación.
\end{itemize}

\subsection{Pantalla de Estadísticas y Resultados}
\begin{itemize}
    \item Visualización de la calificación promedio de cada profesor.
    \item Gráficos o tablas que muestren la distribución de las calificaciones.
\end{itemize}

\subsection{Pantalla de Ayuda/Soporte}
\begin{itemize}
    \item Información sobre el uso de la plataforma.
    \item Preguntas frecuentes y contacto para soporte.
\end{itemize}

\section{Recursos Funcionales y No Funcionales}

\subsection{Requisitos Funcionales}
\begin{itemize}
    \item \textbf{Calificación Anónima:}
    \begin{itemize}
        \item Permitir a los alumnos calificar a los profesores sin identificar al usuario.
        \item La calificación debe ser de 1 a 5 estrellas.
    \end{itemize}
    \item \textbf{Visualización de Profesores:}
    \begin{itemize}
        \item Mostrar un listado de profesores con opción de búsqueda y filtrado.
    \end{itemize}
    \item \textbf{Detalle del Profesor y Calificación:}
    \begin{itemize}
        \item Permitir ver la información básica del profesor y enviar una calificación.
        \item Posibilidad de dejar un comentario opcional.
    \end{itemize}
    \item \textbf{Estadísticas:}
    \begin{itemize}
        \item Calcular y mostrar la calificación promedio y la distribución de las calificaciones.
    \end{itemize}
\end{itemize}

\subsection{Requisitos No Funcionales}
\begin{itemize}
    \item \textbf{Seguridad y Privacidad:}
    \begin{itemize}
        \item Garantizar el anonimato de los alumnos al realizar las calificaciones.
        \item Proteger la integridad de los datos.
    \end{itemize}
    \item \textbf{Usabilidad:}
    \begin{itemize}
        \item Interfaz sencilla e intuitiva.
        \item Diseño responsivo para uso en dispositivos móviles y de escritorio.
    \end{itemize}
    \item \textbf{Rendimiento:}
    \begin{itemize}
        \item Respuesta rápida en la carga de profesores y envío de calificaciones.
    \end{itemize}
    \item \textbf{Mantenibilidad:}
    \begin{itemize}
        \item Código modular y bien documentado para facilitar futuras actualizaciones.
    \end{itemize}
\end{itemize}

\section{Documentación Técnica}

\subsection{Arquitectura del Sistema}
\begin{itemize}
    \item \textbf{Frontend:}
    \begin{itemize}
        \item Aplicación web desarrollada con tecnologías modernas (por ejemplo, React.js, Angular o Vue.js).
    \end{itemize}
    \item \textbf{Backend:}
    \begin{itemize}
        \item API REST para gestionar la obtención de la lista de profesores, envío de calificaciones y consulta de estadísticas.
        \item Tecnologías sugeridas: \texttt{Node.js}, \texttt{Django} o \texttt{Flask}.
    \end{itemize}
    \item \textbf{Base de Datos:}
    \begin{itemize}
        \item Sistema relacional (MySQL, PostgreSQL) para almacenar la información de los profesores y las calificaciones.
    \end{itemize}
    \item \textbf{Servicios Adicionales:}
    \begin{itemize}
        \item Integración de herramientas para visualización de datos (por ejemplo, Chart.js o D3.js) en la sección de estadísticas.
    \end{itemize}
\end{itemize}

\subsection{Casos de Uso Principales}
\begin{itemize}
    \item \textbf{CU1 – Visualización de Profesores:}
    \begin{itemize}
        \item \textbf{Actor:} Alumno.
        \item \textbf{Flujo:} El alumno accede a la lista de profesores, utiliza filtros y selecciona un profesor para ver su detalle.
    \end{itemize}
    \item \textbf{CU2 – Calificación de un Profesor:}
    \begin{itemize}
        \item \textbf{Actor:} Alumno.
        \item \textbf{Flujo:} Desde el detalle del profesor, el alumno selecciona una calificación (1-5 estrellas), puede agregar un comentario opcional y envía la calificación de manera anónima.
    \end{itemize}
    \item \textbf{CU3 – Visualización de Estadísticas:}
    \begin{itemize}
        \item \textbf{Actor:} Cualquier usuario.
        \item \textbf{Flujo:} El usuario consulta las calificaciones promedio y la distribución de las calificaciones de cada profesor.
    \end{itemize}
\end{itemize}

\subsection{Especificación de APIs}
\begin{itemize}
    \item \textbf{Endpoint de Profesores:} \texttt{/api/profesores} (GET para obtener la lista y detalle de profesores).
    \item \textbf{Endpoint de Calificaciones:} \texttt{/api/calificaciones} (POST para enviar una nueva calificación, GET para consultar estadísticas).
\end{itemize}

\subsection{Manual de Instalación y Despliegue}
\textbf{Requisitos Previos:}
\begin{itemize}
    \item Servidores configurados (por ejemplo, AWS, Heroku, etc.).
    \item Variables de entorno para conexión a la base de datos y otros servicios.
\end{itemize}
\textbf{Pasos:}
\begin{enumerate}
    \item Clonar el repositorio del proyecto.
    \item Instalar las dependencias necesarias (por ejemplo, mediante \texttt{npm} o \texttt{pip}).
    \item Configurar las variables de entorno.
    \item Ejecutar las migraciones de la base de datos y los seeds si es necesario.
    \item Desplegar el backend y el frontend.
\end{enumerate}

\subsection{Pruebas y Calidad}
\begin{itemize}
    \item \textbf{Pruebas Unitarias y de Integración:}
    \begin{itemize}
        \item Cobertura de las funcionalidades críticas (envío de calificaciones, obtención de estadísticas).
    \end{itemize}
    \item \textbf{Pruebas de Usabilidad:}
    \begin{itemize}
        \item Testeo con alumnos de la UPV para asegurar la correcta comprensión y uso de la interfaz.
    \end{itemize}
    \item \textbf{Pruebas de Seguridad:}
    \begin{itemize}
        \item Verificar el anonimato en el envío de calificaciones y la integridad de los datos.
    \end{itemize}
\end{itemize}

\section{Cronograma del Proyecto}

A continuación se muestra un ejemplo de cronograma estimado (la duración puede variar según el equipo y alcance):

\begin{center}
\begin{tabularx}{\textwidth}{@{}lp{3cm}X@{}}
\toprule
\textbf{Fase} & \textbf{Duración Estimada} & \textbf{Actividades Clave} \\ \midrule
Análisis y Requerimientos & 1 semana & Reunión inicial, definición de requisitos y alcance. \\
Diseño UI/UX & 2 semanas & Diseño de wireframes, prototipos y validación con usuarios. \\
Desarrollo Frontend & 3 semanas & Implementación de la interfaz, listado de profesores y pantalla de calificación. \\
Desarrollo Backend \& API & 3 semanas & Creación de endpoints para profesores y calificaciones, integración con la base de datos. \\
Integración y Testing & 2 semanas & Pruebas unitarias, de integración y ajustes en el flujo de navegación. \\
Despliegue y Puesta en Producción & 1 semana & Configuración de servidores, despliegue y monitoreo. \\
Documentación y Capacitación & 1 semana & Elaboración de manuales y capacitación a administradores. \\ \bottomrule
\end{tabularx}
\end{center}

\noindent \textbf{Total estimado:} 13 -- 14 semanas.

\section{Diseño de la Base de Datos}

A continuación se presenta un diseño relacional que incluye las siguientes tablas:

\subsection{Tabla \texttt{Profesores}}
\begin{center}
\begin{tabularx}{\textwidth}{@{}p{4cm}p{3cm}X@{}}
\toprule
\textbf{Campo} & \textbf{Tipo de Dato} & \textbf{Descripción} \\ \midrule
id\_profesor   & INT (PK, AutoInc)  & Identificador único del profesor \\
nombre         & VARCHAR(100)       & Nombre del profesor \\
departamento   & VARCHAR(100)       & Departamento o área de enseñanza \\
foto           & VARCHAR(255)       & URL de la foto del profesor \\ \bottomrule
\end{tabularx}
\end{center}

\subsection{Tabla \texttt{Calificaciones}}
\begin{center}
\begin{tabularx}{\textwidth}{@{}p{4cm}p{3cm}X@{}}
\toprule
\textbf{Campo} & \textbf{Tipo de Dato} & \textbf{Descripción} \\ \midrule
id\_calificacion & INT (PK, AutoInc) & Identificador único de la calificación \\
id\_profesor   & INT (FK)           & Referencia al profesor (\texttt{Profesores.id\_profesor}) \\
estrellas      & INT                & Calificación en estrellas (1-5) \\
comentario     & TEXT               & Comentario opcional \\
fecha          & TIMESTAMP          & Fecha y hora de la calificación \\ \bottomrule
\end{tabularx}
\end{center}

\end{document}
